\documentclass[10pt,a4paper]{moderncv}
\moderncvtheme[blue]{casual}                
\usepackage[utf8]{inputenc}
\usepackage[scale=0.8]{geometry}

\firstname{Gaëtan}
\familyname{Lesauvage}
\birthday{11/10/1983}
\title{Concepteur / Développeur}
\address{1 place des Charbonniers}{91430 IGNY}    
\mobile{06 51 66 38 72}                    
\email{gaetanlesauvage@gmail.com}                      
\homepage{http://litis.univ-lehavre.fr/\textasciitilde lesauvage}


\begin{document}
\maketitle

\section{Diplômes et Études}
\cventry{Depuis Septembre 2009}{Préparation d'un Doctorat en Informatique}{LITIS}{Université du Havre}{}
{
\begin{tabular}{p{0cm}p{13cm}}
& Sujet : ``Optimisation dynamique en contexte incertain''\\
& Encadrement : Pr. F. Guinand et Dr. S. Balev\\
& Financement : Bourse ministérielle
\end{tabular}
}

\cventry{2008}{Master Recherche (DEA) Mathématiques et Informatique Appliquées aux Systèmes Complexes}{Université du Havre}{}{}{
\begin{tabular}{p{0cm}p{14cm}}
  & Mention : Très Bien - majeur de promotion\\
  & Stage de recherche : ``Gestion dynamique des activités des chariots cavaliers sur un terminal portuaire à conteneurs en environnement incertain : résolution par intelligence collective''\\
  & Encadrement : Pr. F. Guinand, Pr. D. Olivier et Dr. S. Balev
\end{tabular}
}

\cventry{2007}{Master 1ère année Mathématiques et Informatique}{Spécialité Informatique}{Université du Havre}{}{
\begin{tabular}{p{0cm}p{14cm}}
  & Mention : Bien
\end{tabular}
}

\cventry{2006}{Licence Mathématiques et Informatique}{Spécialité Informatique}{Université du Havre}{}{
\begin{tabular}{p{0cm}p{14cm}}
  & Mention : Bien
\end{tabular}
}

\cventry{2005}{DUT Informatique}{IUT de Caucriauville}{Université du Havre}{}{}

\cventry{2004}{DUT Information et Communication}{IUT du Havre}{Université du Havre}{}{}

\cventry{2002}{Baccalauréat série Économique et Sociale}{Lycée Guy de Maupassant}{Fécamp}{}{
\begin{tabular}{p{0cm}p{14cm}}
  & Mention : Assez Bien
\end{tabular}
}
\section{Expériences}
\cventry{2008 - 2009}{Ingénieur de recherche}{LITIS}{Université du Havre}{}{Attaché au projet CALAS}
\cventry{2006}{Stage de 4 mois en développement}{Millenium Chemical}{Le Havre}{}{
Étude et développement d'applications Access-Excel/VBA pour la gestion qualité de la documentation du site et développement d'une base de données associée
}
\cventry{2005}{Stage de 10 semaines en développement}{Total, Raffinerie de Normandie}{Gonfreville}{}{
Conception et développement d'applications Excel/VBA pour la planification et la gestion des stocks d'huiles et de paraffines. Rédaction des manuels d'administration et d'utilisation
}
\cventry{Juillet – Aout 2004}{Chargé de communication}{ATOFINA}{Gonfreville}{}{
Conception de divers supports de présentations et de communication dans le cadre de l'optimisation des tâches courantes de l'unité. Aide à la conception d'outils informatiques de statistiques pour les unités de maintenance}
\cventry{Avril – Juin 2004}{Stage en communication de 9 semaines}{ATOFINA}{Gonfreville}{}{
Conception d'un support de communication de type PowerPoint afin d'optimiser le processus d'accueil au poste de l'Unité Opérationnelle Énergie
}

\section{Enseignements}
\cventry{2011-2012}{C2i (Certificat Informatique et Internet)}{Licence 1}{UFR ST}{Université du Havre}{}
\cventry{2010-2012}{Algorithmie et JAVA}{Licence 2}{UFR ST}{Université du Havre}{}
\cventry{2009-2011}{Web dynamique}{Master 2}{UFR ST}{Université du Havre}{}
\cventry{2008-2009}{Algorithmie et JAVA}{Info 1}{Département Informatique}{IUT de Caucriauville}{}
\cventry{}{Algorithmie avancée et JAVA}{Info 1}{Département Informatique}{IUT de Caucriauville}{}
\cventry{}{C2i (Certificat Informatique et Internet)}{Licence 1}{UFR ST}{Université du Havre}{}
\cventry{2006-2008}{Tutorat d'Algorithmie et de Programmation Objet}{Info 1 et Année Spéciale}{Département Informatique}{IUT de Caucriauville}{}
\cventry{Déc 2006-fév 2007}{Tutorat de JAVA}{Licence Professionnelle}{Département Informatique}{IUT de Caucriauville}{}

\section{Publications}
\cventry{2011}{D$^2$CTS : A Dynamic and Distributed Container Terminal Simulator}{Lesauvage G., Balev S. and Guinand F.}{The 14th International Conference on Harbor, Maritime \& Multimodal Logistics Modelling and Simulation}{Roma, Italy}{}
\cventry{2011}{Routage des chariots cavaliers sur un terminal portuaire à conteneurs}{Lesauvage G., Balev S. and Guinand F.}{ROADEF 2011}{Saint-Etienne, France}{}
\cventry{2010}{Planification en environnement incertain : application à la gestion d'un terminal portuaire à conteneurs}{Lesauvage G.}{ROADEF 2010}{Toulouse, France}{}
\cventry{2009}{Gestion dynamique des activités des chariots cavaliers sur un terminal portuaire à conteneurs en
environnement incertain : approche par intelligence collective}{Lesauvage G.}{On-line proceedings of conference Majestic2009}{University of Avignon, France}{}{}
\cventry{}{Dynamical handling of straddle carriers activities on a container terminal in uncertain environment - a
swarm intelligence approach}{S. Balev, F. Guinand, G. Lesauvage, and D. Olivier}{The 3rd International Conference on Complex Systems and Applications}{University of Le Havre, France}{}{}

\section{Projets Réalisés}
\cventry{2008-2012}{D$^2$CTS : A Dynamic and Distributed Container Terminal Simulator}{Modélisation et implémentation en JAVA d'un simulateur capable de modéliser à la fois la structure et la dynamique d'un terminal portuaire à conteneurs. Les calculs sont distribués grâce à la technologie RMI de JAVA}{}{}{}
\cventry{2007-2008}{Systèmes Multi-Agents}{Développement en JAVA d'une plate-forme de simulation générique et de 3 applications : un jeu de la vie de Conway, une coloration de graphe et un modèle proies-prédateurs à 3 espèces}{}{}{}
\cventry{}{Graphes dynamiques}{modélisation et implémentation JAVA d'un réseau mobile ad-hoc}{}{}{}
\cventry{}{Bio-informatique}{implémentation C de l'algorithme de Smith-Waterman permettant de déterminer les meilleurs alignements de séquences d'ADN}{}{}{}
\cventry{}{Optimisation Combinatoire}{implémentation JAVA d'un algorithme d'optimisation de flot maximal}{}{}{}
\cventry{2006-2007}{Mémoire de fin d'année}{Programmation Génétique : comparaison avec les algorithmes de colonies de fourmis}{}{}{}
\cventry{}{Intelligence Artificielle}{implémentation JAVA d'un algorithme de routage sur un robot Lego MindStorms NXT}{}{}{}
\cventry{}{Intelligence Artificielle}{implémentation JAVA de l'algorithme A* et application au jeu de Gauntlet}{}{}{}
\cventry{}{Parallélisme}{implémentation C d'une méthode de Branch and Bound distribuée grâce à la librairie MPI}{}{}{}
\cventry{}{Infographie}{implémentation C/OpenGL (Mesa/Glut) 3D d'un Rubik's Cube}{}{}{}
\cventry{}{Informatique théorique}{implémentation JAVA d'une machine de Turing}{}{}{}
\cventry{}{Cryptographie}{implémentation JAVA de l'algorithme de chiffrage DES}{}{}{}
\cventry{}{Réseau}{implémentation JAVA de l'algorithme de codage de Hamming}{}{}{}
\cventry{2005-2006}{Génie Logiciel}{modélisation et implémentation JAVA du système de planification de production d'une PME manufacturière}{}{}{}
\cventry{2004-2005}{Bases de Données}{implémentation JAVA/PostGreSQL d'un logiciel de gestion de dvdthèque}{}{}{}
\cventry{}{Réseaux}{implémentation JAVA d'un serveur multithreads}{}{}{}
\cventry{}{Projet}{modélisation et implémentation JAVA d'une plate-forme de simulation logicielle}{}{}{}
\cventry{}{Langage C}{implémentation C du jeu de puissance 4}{}{}{}

\section{Compétences Informatiques}
\cvline{Langages}{Java, C++, C, CLISP, CAML, Bash Shell}{}{}{}{}
\cvline{Conception}{UML 2, Merise}{}{}{}{}
\cvline{SGBD}{PostgreSQL, MySQL, Access}{}{}{}{}
\cvline{Infographie}{OpenGL (MESA/GLUT) et Java3D}{}{}{}{}
\cvline{Web}{XHTML/CSS, XML, Javascript, J2EE (JDBC, RMI, Web Services, EJB, Servlets, Portlets, JSP),PHP}{}{}{}{}
\cvline{Autre}{Maîtrise de \LaTeX, connaissances en architecture système et réseau. Maîtrise des OS à base GNU/Linux}{}{}{}{}

\section{Langues}
\cvlanguage{Anglais}{lu, parlé, écrit}{}
\cvlanguage{Allemand}{notions}{}
\cvlanguage{Roumain}{notions}{}


\section{Centres d'intérêt}
\cvline{Loisirs}{Guitare, Chant, Badminton}
\cvline{Associatif}{Secrétaire adjoint du Poona Montivilliers Badminton Club\newline{}Ancien représentant des doctorants au conseil de laboratoire du LITIS}

\section{Autres}
\cvline{Mobilité}{permis B}{}

\end{document}