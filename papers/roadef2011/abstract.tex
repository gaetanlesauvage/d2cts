\documentclass{roadef2011}

\usepackage{pslatex}

%opening
\title{qdqsdqsdq}
\author{sdqsdqsd}

\begin{document}



\begin{abstract}
Un terminal portuaire � conteneurs est un syst�me complexe ouvert compos� de plusieurs entit�s en interactions. Divers engins de manutention, comme les chariots cavaliers, permettent de d�placer les conteneurs au sein du terminal afin de r�pondre le plus efficacement possible aux demandes des navires, trains ou camions en attente de chargement ou de d�chargement. Nous nous int�ressons ici � la gestion de ces d�placements. En effet, le graphe routier d'un terminal � conteneurs contient des arcs repr�sentant les routes du terminal et des arcs FIFO repr�sentant les trav�es de conteneurs. Les chariots cavaliers ne pouvant pas se croiser dans une trav�e, il faut prendre en compte les temps d'attente en entr�e de trav�es dans le calcul des itin�raires des v�hicules. Le graphe routier du terminal est donc dynamique, le temps de parcours d'un arc d�pend de la distance � parcourir, et � la fois de la vitesse du v�hicule et du temps d'attente au moment de la travers�e. Le terminal contient plusieurs v�hicules et il faut donc optimiser la totalit� des itin�raires. Nous proposons deux m�ta-heuristiques de r�solution. Tout d'abord un algorithme m�m�tique bas� sur un codage simple du probl�me, puis un algorithme fourmis � plusieurs colonies comportant un m�canisme de comp�tition entre les colonies de fourmis.
\end{abstract}

\section{}

\end{document}
