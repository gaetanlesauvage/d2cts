% Exemple d'utilisation de la classe roadef2011 pour le congrès ROADEF 2011
\documentclass{roadef2011}

\usepackage{pslatex}

\begin{document}


% Le titre du papier
\title{Routage des chariots cavaliers sur un terminal portuaire à conteneurs}

% Le titre court
\def\shorttitle{Titre court}

% Les auteurs et leur numéro d'affiliation
\author{Gaëtan Lesauvage, Stefan Balev, Frédéric Guinand}


% Les affiliations (par ordre croissant des numéros d'affiliation) séparées par \and
\institute{
LITIS, Université du Havre\\ 25 rue Ph. Lebon, BP 540\\76058 Le Havre Cedex, France \\
\email{ gaetanlesauvage@gmail.com, \{stefan.balev,frederic.guinand\}@univ-lehavre.fr}
}


% Création de la page de titre
\maketitle
\thispagestyle{empty}

% Les mots-clés
\keywords{problème de plus court chemins, graphe dynamique, terminal à conteneurs, optimisation multi-critères}

\section{Introduction}

Un terminal portuaire à conteneurs est un système complexe ouvert composé de plusieurs entités en interaction. Divers engins de manutention permettent de déplacer les conteneurs au sein du terminal afin de répondre le plus efficacement possible aux demandes des navires, trains ou camions en attente de chargement ou de déchargement. 

Dans le soucis d'améliorer la qualité de service fourni au clients et de réduire le coût d'exploitation du terminal, l'étude des déplacements des véhicules s'avère d'une grande importance. 

\section{Problème}
%Deplacements des chariots pour des missions
La gestion d’un terminal portuaire à conteneurs relève de plusieurs problèmes d’optimisation. Le but étant de minimiser les temps d’attente des clients tout en minimisant les coûts d’exploitation du terminal, un soin tout particulier est apporté à l’affectation des missions aux chariots cavaliers. De plus, afin de réduire le temps d’exécution de ces missions, il est également nécessaire de disposer les conteneurs de façon efficace.
Les chariots cavaliers ont pour mission de déplacer des conteneurs à l'intérieur du terminal. Chaque mission comporte deux fenêtres de temps : l'une pour la collecte du conteneur et l'autre pour la livraison. Le non respect de ces fenêtres de collecte et de livraison entraînne un coût supplémentaire au terminal. Les routes des chariots cavaliers doivent donc prendre en compte ces fenêtres de temps.

%Temps d'attente
Sur un terminal portuaire à conteneurs, les itinéraires possibles sont relativement réduits mais les chariots ne peuvent se croiser que sur les routes. En effet, ils enjambent les conteneurs stockés dans les travées et ne peuvent donc pas se croiser à l'intérieur de ces travées. Ces blocages peuvent causer des retards importants pour la réalisation d'une mission. Notre but est donc de définir les itinéraires des chariots cavaliers prennant en compte les temps d'attente en entrée de travées afin à la fois de minimiser la durée totale de parcours (déplacement + blocage), le coût de déplacement (distance) et la date d'arrivée du chariot (respect de la fenêtre de temps).

%Graphe routier
\section{Graphe routier}
Le graphe routier du terminal est donc un graphe partiellement FIFO, c'est-à-dire que les arcs modélisant les travées sont FIFO alors que les arcs modélisant les routes ne sont pas FIFO. De plus la durée de parcours d'un arcs dépend à la fois de la distance à parcourir et du temps. En effet, la durée de parcourt d'un arc dépend directement de la vitesse du chariot (chariot plein ou vide) et également du traffic sur l'arc au moment de la traversée. On défini donc la durée de parcours de l'arc $(i,j)$ par cette formule : 
	\begin{equation}
	  duree(i, j , t, v) = (distance(i,j) / vitesse(v, i, j, t)) + attente(i,j,t)
	\end{equation}

\subsection{Complexit\'e}
Le problème du plus court chemin en temps est connu pour etre résolu en temps polynomial pour des graphes FIFO\cite{Orda1990}.  En revanche, le problème de plus court chemin en cout est NP-difficile\cite{Ahuja2003}. Le problème ici étudié est un problème de plus court chemin en cout sur un graphe FIFO.

\subsection{Etat de l'art}

Shortest Path Problem with Time Windows and Shortest Path Problem with waiting costs. Shortest path in delay time dependent networks.

(Dynamic algorithms : incremental, decremental, fully dynamic)
FIFO Networks – non FIFO Networks
Waiting models : Unrestricted waiting, Forbidden waiting, Source Waiting
Minimum cost walk problem
Problem complexity

\section{Algorithme de routage}

Optimisation locale des routes des chariots en tenant compte des routes des autres véhicules précédemment établies.
Comment rendre cette optimisation globale ? Lors de la planification d'une route, il faudrait également recalculer les routes des autres véhicules en prenant comme origine leur position actuelle. Puis recommencer dans un autre ordre. => O(n!). 
Proposition d'un algorithme permettant la résolution du problème dans un temps acceptable ?
Proposition d'un algorithme méta-heuristique ?

	
\section{R\'esultats et perspectives}


% La bibliographie
\bibliographystyle{plain}
\bibliography{biblio}

\end{document}
