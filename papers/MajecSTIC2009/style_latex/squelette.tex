%
% Fichier exemple pour MajecSTIC 2009
% -----------------------------------
% Par le comit� de pilotage MajecSTIC
% majecstic-pilotage@irisa.fr
% 
% Vous pouvez �diter ce fichier pour composer votre article. Respectez la 
% langue fran�aise, pour vous aider ce document comporte des consignes 
% typographiques ainsi que des conseils pour la composition des figures et 
% des algorithmes.
%
%
%
%%%%%% NE PAS MODIFIER
%
% gillemets a la francaise
\def\leftnote#1{\leavevmode\vadjust{\setbox1=\vtop{\hsize 20mm
	\parindent=0pt\small\baselineskip=9pt
	\rightskip=4mm plus 4mm#1}
	\hbox{\kern-2cm\smash{\box1}}}}
% encore quelques petits symboles particuliers
	\font\myl=manfnt
	\def\panneau{{\myl\char"7F}}
	\def\boxone{{\myl\char"1C}}
	\def\boxtwo{{\myl\char"1D}}
	\def\ortf{{\myl\char"1E}}
	\def\fleurone{{\myl\char"26}}
	\def\fleurtwo{{\myl\char"27}}
	\def\diams{{\myl\char"23}}
	\def\cible{{\myl\char"24}}
	\def\carre{{\myl\char"25}}
	\def\fleche{{\myl\char"79}}
	\def\panneaubis{{\myl\char"7E}}
\def\panneauinverse{{\myl\char"00}}

% Conserver ces commandes (Debut)
\documentclass[twoside,a4paper,10pt]{article}
\usepackage[T1]{fontenc}
\usepackage[latin1]{inputenc}
\usepackage[frenchb]{babel}
\usepackage{majecstic2009,euler,palatino}
\usepackage[french,ruled,vlined,linesnumbered]{algorithm2e}
\dontprintsemicolon
\Setnlskip{0.5em}
\incmargin{1.2em}
\usepackage{epsfig,shadow}
\usepackage{amsmath}
\usepackage{amssymb}
\usepackage{url}
\usepackage{cite}
\usepackage{vmargin}
\pagestyle{myheadings}
\setpapersize{A4}
% Conserver ces commandes (Fin)

%===========================================================
%                               Title
%===========================================================
\newcommand{\filet}{\noindent\rule[0mm]{\textwidth}{0.2mm}}

\toappear{1} % Conserver cette ligne pour la version finale

\begin{document}

\parindent=0pt

% � FAIRE modifier en ajoutant votre titre.
% Le titre courant peut-�tre trop long, dans ce cas indiquez un titre 
% plus court dans la commande shorttitle (sinon, indiquez le m�me)
\title{\Large\bf Le titre de votre article}

% � FAIRE modifier en ajoutant dans le premier champ vos noms et dans votre
% titre. Attention, il faut que ca tienne sur une largeur de page, donc il 
% peut etre necessaire de donner un titre plus court, ou de ne mettre que les
% noms de famille des auteurs
% Exemples:
% \markboth{Josiane Balasko, Muriel Robin \& Yves Montand}{Nos ann�es palace.}
% \markboth
% {Dupont, Dupond, Haddock, Milou, Tintin \& Tournesol}
% {De l'exploration des profondeurs � la conqu�te spatiale}
\markboth
{NOMS}
{TITRE}

% � FAIRE Indiquez vos noms ici. Nom1, Nom2 et Nom3. Utilisez $^i$ pour 
% l'adresse i.
\author{Pr�nom Nom$^1$\thanks{Remarque �ventuelle.}
    et Pr�nom Nom$^2$}

% � FAIRE Indiquez vos addresses ici. Attention, la premi�re addresse commence
% juste apr�s le { de la commande.
\address{1: Universit� de Quelque Part, Institut Machin, 30 impasse de l'anglicisme, 12345 Ville - France.\\
2: Une autre adresse.\\ 
~\\
Contact: \texttt{mon.mail@universite.fr}
}

\maketitle

%===========================================================         %
%R\'esum\'e
%===========================================================  

\Resume{Votre r�sum� ici, 15 lignes.}

\Abstract{Your abstract should be no longer than 15 lines and contain no references.}

\MotsCles{un maximum de 5 mots significatifs, en fran�ais, doivent �tre 
	isol�s sous forme de mots-cl�s.}
	% 5 mots cl\'es seulement!!!
\Keywords{One, two, three, four, five}
%=========================================================
\section{Introduction}
%=========================================================

Voulez-vous \og{}\emph{vraiment}\fg{} citer~\cite{bogdanoff:these,sonia:point} ?

\bibliographystyle{fplain}
\bibliography{references}

% � FAIRE: remerciements (ou supprimez les deux lignes suivantes)
{\filet \small\\
{\em Vous pouvez placer ici des remerciements}\\ \filet } 


\newpage
j

\newpage
j

\newpage
jj
jj


\end{document}
% fin








