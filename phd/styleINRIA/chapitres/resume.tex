\thispagestyle{empty}

\begin{center}
\textbf{\large{Résumé}}
\end{center}

\fontsize{11pt}{1}\selectfont

Dans cette étude, nous abordons le diagnostic des défauts
rotoriques dans les machines asynchrones à cage d'écureuil. Après
avoir décrit les différents éléments de constitution d'une machine
asynchrone et les principaux défauts pouvant survenir sur ceux-ci,
nous proposons un modèle de machine basée sur la méthode des
circuits électriques magnétiquement couplés. Ce modèle permet
d'étudier l'influence d'un défaut de barre sur le comportement
général du moteur asynchrone. En complément de l'étude menée, nous
mettons en évidence l'importance de l'analyse des harmoniques
d'espace pour le diagnostic des défauts rotoriques.

L'étude des phénomènes créés par la présence d'un défaut rotorique
sur les différentes grandeurs temporelles de la machine, nous nous
intéressons plus particulièrement au développement de nouvelles
méthodes de diagnostic. Nous présentons trois méthodes permettant
la détection d'un défaut rotorique. La première méthode s'appuie
sur l'évaluation de plusieurs indices calculés à partir de
l'amplitude des composantes présentes dans les spectres de la
puissance instantanée et du courant absorbé par le moteur. Les
résultats obtenus avec cette approche permettent de détecter la
présence d'un défaut naissant (une barre partiellement cassée)
lorsque le couple de charge est supérieur ou égal à 10\% du couple
nominale ainsi qu'une barre complètement cassée lorsque le moteur
fonctionne à vide.

La seconde méthode de détection proposée utilise la phase du
spectre du courant statorique calculée à partir d'une transformée
de Fourier. Cette approche a la particularité de ne se baser sur
aucun seuil de référence pour établir la présence d'une rupture de
barre au sein de la cage d'écureuil. Avec cette approche, nous
avons pu détecter la présence d'une barre rotorique complètement
cassée. Malheureusement, le bruit important contenu dans ce signal
ne permet pas de détecter un défaut rotorique naissant. Pour
pallier ce problème, nous utilisons la phase du signal analytique
obtenue par une transformée de Hilbert du module du spectre du
courant statorique. Cette nouvelle approche, qui permet d'obtenir
un signal plus stable et moins bruité, permet la détection d'une
barre partiellement cassée et d'une barre totalement cassée pour
une charge supérieure ou égale à 25\%.

\vspace{0.25cm}

\noindent\textbf{Mots-clés~:} Moteur asynchrone, Modélisation,
Harmoniques d'espace, Diagnostic, Rupture de barre, Indice de
modulation, Périodogramme de Bartlett, Transformée de Fourier,
Transformée de Hilbert. \fontsize{12pt}{1.5}\selectfont

\vskip15mm


\begin{center}
\textbf{Modelisation and diagnosis of induction machine in
presence of failures}
\end{center}

%\vskip5mm

\begin{center}
\textbf{\large{Abstract}}
\end{center}

\fontsize{10pt}{1}\selectfont

In this study, we move on to the broken rotor bar diagnosis of
squirrel-cage induction machines. The first part is devoted to the
development of a model which is based on the magnetically coupled
electric circuits. This type of modelling makes it possible to
study the influence of a bar defect on the general behavior of the
asynchronous motor. In complement of the undertaken study, we
underscore the importance of the analysis of the space harmonics
for the broken rotor bar diagnosis.

After having studied the phenomena created by the presence of a
rotor defect on the various temporal sizes of the induction
machine, we turn a particularly attention in the development of
new diagnosis methods. We present three methods allowing detection
of a rotor defect of an induction machine.

The first method is based on the evaluation of several indexes
calculated starting from the amplitude of the components present
in the spectra of the instantaneous power and the line current.
The results obtained with this new approach make it possible to
detect an incipient defect (a partially broken bar) when the
asynchronous motor works with a load torque higher or equal to
10\% of the nominal torque as well as a completely broken bar when
the motor works without load.

The second method of detection suggested uses the stator current
spectrum phase calculated starting from a Fourier Transform. This
approach has the characteristic to be based on any threshold of
reference to establish the presence of a broken rotor bar, which
is usually necessary to detect this type of defect. The validation
of this method on various experimental tests makes it possible to
detect the presence of one broken bar with a minimum load torque
of 25\%. Unfortunately, the important noise contained in this
signal does not make it possible to detect an incipient rotor
defect. To get round this problem, we use the analytic signal
phase calculated starting from the Hilbert transform of the stator
current spectrum modulus. This new approach, which makes it
possible to obtain a more stable and less disturbed signal, makes
it possible to diagnose a partially broken bar and one broken bar
for a load torque superior or equal to 25\%.

\vspace{0.25cm}

\noindent\textbf{Key-words~:} Induction motor, Modelling, Space
Harmonics, Diagnosis, Broken rotor bar, Modulation index, Bartlett
periodogram, Fourier Transform, Hilbert Transform.
\fontsize{12pt}{1.5}\selectfont
