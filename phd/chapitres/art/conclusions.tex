Dans ce premier chapitre nous avons passé en revue tous les domaines, tous les principes et toutes les méthodes de résolution touchant à des degrés différents le problème d'optimisation traité dans cette thèse.

Dans la première section nous avons introduit la notion d'optimisation dynamique sous incertitude. Le cadre de l'incertitude dans cette thèse a été défini et nous ne nous intéresserons uniquement aux systèmes réactifs où la méthode de résolution elle même sera adaptée à la dynamicité et à l'incertitude du problème.

Dans la seconde section nous avons introduit le problème du voyageur de commerce ainsi que ses diverses variantes. Ce problème est à la base des autres problèmes d'optimisation évoqués dans cette thèse. Nous avons dressé un état de l'art des méthodes les plus couramment utilisées afin de résoudre le problème de voyageur de commerce de façon exacte ou approchée et en détaillant dans ce dernier cas les méthodes de construction, d'amélioration et les métaheuristiques. La version généralisée du problème à $m$ voyageurs a également été étudiée et correspond au problème de tournées de véhicules dans lequel la capacité et la distance ne sont pas contraints. Nous avons montré qu'il existe deux paradigmes de résolution du problème : d'une part résoudre $m$ problèmes de voyageur de commerce, et d'autre part résoudre le problème classique issu de la transformation du problème multiple. Là aussi plusieurs méthodes de résolutions existes. Concernant les méthodes exactes, des méthodes de \textit{Branch and Price} et de \textit{Branch and Bound} permettent de résoudre des instances de 120 villes pour 2 à 12 voyageurs. Les méthodes approchées permettent de résoudre des instances plus importantes grâce à des méta-heuristiques (recherche tabou, recuit simulé, réseaux de neurones artificiels, algorithmes génétiques et algorithmes fourmis). 

Dans la troisième section nous avons présenté le problème de tournées de véhicules. Dans sa forme la plus classique il consiste à optimiser les tournées d'une flotte de véhicules de façon à servir chaque client en respectant les capacités des véhicules (ainsi que bien souvent une distance maximale par véhicule) tout en minimisant la distance totale parcourue. Certaines versions du problèmes comportent des contraintes temporelles sous forme de fenêtres de temps. Dans sa version dynamique, le problème devient plus difficile à résoudre à cause du besoin d'obtenir une solution rapidement. L'incertitude liée à la dynamique impose l'utilisation de métaheuristiques permettant d'utiliser les solutions calculées précédemment pour calculer des nouvelles solutions lorsqu'un événement survient. Les méthodes les plus utilisées sont la recherche tabou, les algorithmes génétiques et les algorithmes fourmis.

Enfin, dans la quatrième partie nous avons passé en revue les caractéristiques des problèmes d'ateliers et plus précisément du problème de \textit{Job Shop}. Tout comme pour les problèmes de voyageurs de commerce et de tournées de véhicules des méthodes de résolution exacte et approchées ont été utilisées. D'ailleurs, Beck et al. ont montré dans \cite{Beck2003} que les problèmes de \textit{Job Shop} et de tournées de véhicules étaient similaires. Néanmoins, tout comme pour les problèmes de tournées de véhicules, seules les méthodes approchées sont utilisées dans les problèmes concrets à cause de la complexité du problème et des délais relativement cours rencontrés dans les environnements de production réels.
Ainsi les méthodes à base de règles de priorité, l'heuristique du goulot d'étranglement ainsi que les métaheuristiques du recuit simulé, de recherche tabou, les algorithmes génétiques et les algorithmes fourmis sont utilisés pour résoudre les instances de taille importante du JSSP.
Concernant la version dynamique du problème, les algorithmes génétique constituent la méthode la plus efficace permettant de proposer une nouvelle planification sans tout recalculer de zéro. Les algorithmes fourmis restent néanmoins une voie à explorer pour résoudre la version dynamique des problèmes d'ateliers et c'est pourquoi ils font l'objet de cette thèse.