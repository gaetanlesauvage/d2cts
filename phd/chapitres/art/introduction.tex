Ce chapitre présente le contexte théorique de cette thèse. Il définit le cadre ainsi que les bases théoriques de l'étude et dresse un état de l'art des travaux en relation avec le problème d'affectation et de routage des chariots cavaliers sur un terminal à conteneurs. 

Ce dernier est effectivement lié à plusieurs autres problèmes abondamment traités dans la littérature et qui seront présentés dans les quatre sections de ce chapitre. L'optimisation dynamique sous incertitude fait l'objet de la première section. La deuxième section sera consacrée au problème du voyageur de commerce et à sa généralisation à $m$ voyageurs. La troisième section sera dédiée aux problèmes de tournées de véhicules. Enfin, la dernière section établira un tour d'horizon des problèmes d'ateliers et en particulier du problème de \textit{Job-Shop}.