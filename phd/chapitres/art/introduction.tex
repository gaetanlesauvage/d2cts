Ce chapitre sert d'introduction et de contextualisation théorique de cette thèse. Il sera question de définir le cadre ainsi que les bases théoriques de l'étude. 

Le problème d'affectation et de routage des chariots cavaliers sur un terminal à conteneurs peut être modélisé par plusieurs problèmes biens connus dans la littérature. La première section sera donc consacrée à l'optimisation dynamique sous incertitude. Puis, la deuxième section est consacrée au problème de voyageur de commerce et à sa généralisation à $m$ voyageurs. La troisième section sera dédiée aux problème de tournées de véhicules. Enfin, la dernière section établira un tour d'horizon des problèmes d'ateliers et en particulier du problème de \textit{Job-Shop}.